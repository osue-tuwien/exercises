% TeX source file
% Sysprog SS 2005
% Beispiel 3: proxy
% Christian Steiner
\input{../../template.ltx}

\begin{document}

\osuetitle{2}

\section*{Aufgabenstellung}

Implementieren Sie ein Programm, welches ein auf der Kommandozeile angegebenes
Programm in einem Kindprozess ausführt und je nach angegebenen Optionen
\osueglvar{stdin}, \osueglvar{stdout} und \osueglvar{stderr} in Dateien umlenkt.

\begin{verbatim}
    SYNOPSIS:
        proxy [-i infile] [-o outfile] [-e errfile] <cmd> [options]
\end{verbatim}

Wird ein Ziel nicht angegeben (eine oder mehrere der Optionen \osuearg{-i},
\osuearg{-o} oder \osuearg{-e} fehlt) oder wurde \osuearg{-} als Dateiname
angegeben, soll der entsprechende Ein-/Ausgabekanal unverändert bleiben. Der
Rückgabewert des \osueprog{proxy}-Programms soll dem Rückgabewert des
ausgeführten Programms entsprechen.

\subsection*{Anleitung}

Verwenden Sie \osuefunc{getopt(3)} zur Argumentbehandlung. Benötigt das
auszuführende Programm eine oder mehrere Optionen, können Sie durch Voranstellen
von \osuearg{--} eine Fehlermeldung von \osuefunc{getopt} vermeiden -- das
Parsen der Parameter endet dann an dieser Stelle.\\
(Beispiel: \osuecmd{./proxy -o outfile -- ls -o *})

Erzeugen Sie anschließend mit \osuefunc{fork(2)} einen Kindprozess. Lenken Sie
mittels \osuefunc{dup2(2)} die entsprechenden Filedeskriptoren um und führen Sie
zuletzt das Programm samt Optionen mit Hilfe von \osuefunc{execvp(2)} aus.
Vergessen Sie im Vaterprozess nicht, den Exit-Status des Kindprozesses abzuholen
(und so die Entstehung eines Zombieprozesses zu vermeiden).

\subsection*{Testen}

Verwenden Sie \osuecmd{proxy}, um mehrere Programme auszuführen:

\begin{verbatim}
	./proxy cat

	./proxy -i Makefile -o Makefile.cat cat

	./proxy -i - -o - -- grep -ni
\end{verbatim}

\osueguidelinestwo

\end{document}
