% TeX source file
% Sysprog SS 2005
% Beispiel 3: proxy
% Christian Steiner
\input{../../template.ltx}

\begin{document}

\osuetitle{2}

\section*{Aufgabenstellung}

Implementieren Sie ein Programm, welches ein auf der Kommandozeile angegebenes
Programm in einem Kindprozess ausführt und je nach angegebenen Optionen 
\emph{stdin}, \emph{stdout} und \emph{stderr} in files umlenkt.

\begin{verbatim}
    SYNOPSIS:
       proxy [-i infile] [-o outfile] [-e errfile] <cmd> [options]
\end{verbatim}

Wird ein Ziel nicht angegeben (eine oder mehrere der Optionen \verb_-i_,
\verb_-o_ oder \verb_-e_ fehlt) oder wurde \verb_-_ als Filename angegeben, soll
der entsprechende Ein-/Ausgabekanal unverändert bleiben. Der Rückgabewert des
\verb_proxy_-Programms soll dem Rückgabewert des ausgeführten Programms
entsprechen.

\section*{Anleitung}

Verwenden Sie \emph{getopt(3)} zur Argumentbehandlung. Benötigt das
auszuführende Programm eine oder mehrere Optionen, können Sie durch Voranstellen
von \verb|--| eine Fehlermeldung von {\em getopt} vermeiden – das Parsen der
Parameter endet dann an dieser Stelle.\\
(Beispiel: \verb|./proxy -o outfile -- ls -o *|)

Erzeugen Sie anschließend mit \emph{fork(2)} einen Kindprozess. Lenken Sie
mittels \emph{dup(2)} oder \emph{dup2(2)} die entsprechenden Filedeskriptoren um
(vergessen Sie nicht, den ursprünglichen Deskriptor gegebenenfalls vorher zu
schließen) und führen Sie zuletzt das Programm samt Optionen mithilfe von
\emph{execvp(2)} aus. Vergessen Sie im Vaterprozess nicht, den Exitstatus des
Kindprozesses abzuholen (und so die Entstehung eines Zombieprozesses zu
vermeiden).

\section*{Testen}

Verwenden Sie \verb_proxy_, um mehrere Programme auszuführen:

\begin{verbatim}
	./proxy cat
	
	./proxy -i Makefile -o Makefile.cat cat

	./proxy -i - -o - -- grep -ni
\end{verbatim}

\osueguidelinestwo

\end{document}
