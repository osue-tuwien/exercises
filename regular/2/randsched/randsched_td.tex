\input{../../template.ltx}

\begin{document}

\osuetitle{2}

\section*{Aufgabenstellung -- randsched}

Sie sind der neue Systemadministrator im lokalen Atomkraftwerk. Schon bald ist
Ihnen klar, dass im Wesentlichen einfach nur alle paar Minuten ein Programm
aufgerufen werden muss, dessen Output in eine Protokolldatei abzutippen ist.
Falls doch mal ein Fehler auftritt, muss aber von Ihnen der sofortige
Reaktor-Shutdown eingeleitet werden. Sich Ihrer kritischen, dennoch gefährlichen
und schlecht bezahlten Position bewusst, fassen Sie schnell den Entschluss,
diesen Vorgang möglichst transparent zu automatisieren.

\subsection*{Anleitung}
Implementieren Sie die folgenden Hilfsprogramme, um den Reaktor zu simulieren:

\begin{verbatim}
    SYNOPSIS:
        rventgas

    SYNOPSIS:
        rshutdown
\end{verbatim}

Das Programm \osueprog{rventgas} soll mit einer Wahrscheinlichkeit von 6/7
erfolgreich mit der Meldung \osueoutput{STATUS OK} terminieren, sonst soll
\osueoutput{PRESSURE TOO HIGH - IMMEDIATE SHUTDOWN REQUIRED} ausgegeben werden
und die Terminierung erfolgt mit einem Fehlercode.

Das Programm \osueprog{rshutdown} schreibt mit einer Wahrscheinlichkeit von 1/3
die Meldung\\
\osueoutput{SHUTDOWN COMPLETED} auf \osueglvar{stdout} und terminiert, sonst
soll \osueoutput{KaBOOM!} ausgegeben und ein Fehlercode zurückgegeben werden.

Nun zum eigentlichen Teil der Aufgabe: Implementieren Sie das folgende Programm:

\begin{verbatim}
    SYNOPSIS:
        schedule [-s <seconds>] [-f <seconds>] <program> <emergency> <logfile>

        -s <seconds>   Zeitfenster Anfang (Default: 1 Sekunde)
        -f <seconds>   Zeitfenster Ende (Default: 0 Sekunden)
        <program>      Programm inkl. Pfad, das wiederholt ausgefuehrt werden soll
        <emergency>    Programm inkl. Pfad, das im Fehlerfall ausgefuehrt wird
        <logfile>      Pfad zu einer Datei, in der die Ausgabe von <program> sowie
                       Erfolg/Misserfolg von <emergency> protokolliert werden
\end{verbatim}

Das Programm \osueprog{schedule} erzeugt einen Kindprozess, der wiederholt
zufällig irgendwann im spezifizierten Zeitfenster (in $s$ bis $s+f$ Sekunden,
als Granularität genügt eine volle Sekunde) das Programm \osueprog{program} als
Enkelkind startet und auf seine Beendigung wartet. Falls \osueprog{program} mit
einem Fehlercode terminiert, wird einmalig \osueprog{emergency} ausgeführt und
der Kindprozess beendet. Wird z.B.\ \osueprog{schedule} mit \osuearg{-s 5} und
\osuearg{-f 2} aufgerufen, wartet der Kindprozess zunächst zwischen 5 und 7
Sekunden und führt dann \osueprog{program} aus. Terminiert \osueprog{program}
fehlerfrei, wartet der Kindprozess wieder zwischen 5 und 7 Sekunden, bevor er
\osueprog{program} ein weiteres Mal ausführt.

Weiters soll der Kindprozess alle Daten, die der \osueprog{program}-Enkelprozess
auf \osueglvar{stdout} ausgegeben hat, über eine Pipe an den Vaterprozess
schicken. Dieser gibt das Gelesene auf (sein) \osueglvar{stdout} aus und fügt es
gleichzeitig an die angegebene \osuefilename{logfile} an. Ausgaben von
\osueprog{emergency} müssen nicht im Logfile aufscheinen; es genügt zu
protokollieren, ob \osueprog{emergency} ohne Fehlercode terminiert hat. Nachdem
\osueprog{emergency} beendet ist, soll auch \osueprog{schedule} terminieren.
Implementieren Sie außerdem noch eine Signalbehandlung, um \osueprog{schedule}
sauber beenden zu können, ohne auf eine Fehlfunktion des Reaktors hoffen zu
müssen.

\subsection*{Testen}

Das typische Output sollte so ähnlich aussehen:

\begin{osuefmtcode}
$ \osueinput{./schedule -s 2 -f 5 ./rventgas ./rshutdown logfile}
STATUS OK
STATUS OK
STATUS OK
STATUS OK
STATUS OK
PRESSURE TOO HIGH - IMMEDIATE SHUTDOWN REQUIRED
SHUTDOWN COMPLETED.
EMERGENCY SUCCESSFUL

$ \osueinput{cat logfile}
STATUS OK
STATUS OK
STATUS OK
STATUS OK
STATUS OK
PRESSURE TOO HIGH - IMMEDIATE SHUTDOWN REQUIRED
EMERGENCY SUCCESSFUL
\end{osuefmtcode}

Die Meldung \osueoutput{SHUTDOWN COMPLETED} ist eigentlich von
\osueprog{rshutdown} und scheint nicht in der Log-Datei auf, wo aber sehr wohl
\osueoutput{EMERGENCY SUCCESSFUL} anzeigt, dass der \osueprog{emergency}-Prozess
erfolgreich war.

\osueguidelinestwo

\end{document}
