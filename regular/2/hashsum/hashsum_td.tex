% TeX source file
% Sysprog WS 2010
% Beispiel 2: Hashsum
% Sebastian Falbesoner
\input{../../template.ltx}

\begin{document}

\osuetitle{2}

\section*{Aufgabenstellung}

Schreiben Sie ein Programm, welches für alle Dateien in einem angegebenen
Verzeichnis jeweils eine Zeile mit dem dazugehörigen MD5-Hashwert und dem
Dateityp ausgibt. Dafür sollen die Standardtools \texttt{ls} (mit Parameter
\texttt{-1a}), \texttt{md5sum}, und \texttt{file} verwendet werden, indem
diese als Kindprozesse per \texttt{fork()}/\texttt{exec()} aufgerufen
werden.


\begin{verbatim}
SYNOPSIS:
	hashes [-i ignorepraefix] <directory>
\end{verbatim}

Mit dem Parameter \verb_-i_ (“ignore”) werden Dateien, die mit dem
darauffolgenden Muster (nicht case-sensitive!) beginnen, ausgenommen;
so führt ein Aufruf \verb_hashes -i . ~_ dazu, dass alle versteckten
Dateien im Home-Verzeichnis nicht behandelt werden.


\subsection*{Anleitung}
Nachdem mit dem Kindprozess \texttt{ls} der gesamte Inhalt des angegebenen
Verzeichnisses ermittelt und an den Vaterprozess zurückgeben wird, erzeugt man
in einer Schleife jeweils für jede der von \texttt{ls} empfangenen Dateinamen
einen Kindprozess der \texttt{md5sum} mit diesem Namen als Argument ausführt,
gleiches danach für \texttt{file}. Weiters werden die gewonnenen Informationen
jeweils in einer Zeile (Format: \texttt{Dateiname md5summe dateityp})
ausgegeben. Beachten Sie, dass es sich nicht bei allen von \texttt{ls}
gelisteten Einträgen um Dateien handelt (z.B.\ \texttt{.} und \texttt{..}); in dem Fall liefert dann
\texttt{md5sum} einen nicht-erfolgreichen Return-Code zurück (welcher
mittels \texttt{wait()/waitpid()} ermittelt werden kann) und der Eintrag soll in
diesem Fall ignoriert werden.

\subsection*{Beispielaufruf}
\begin{verbatim}
[~]$ mkdir hashtests
[~]$ cd hashtests
[hashtests]$ touch emptyfile
[hashtests]$ echo "das ist das haus vom nikolaus" > textfile
[hashtests]$ echo "compressme" | gzip -cf > compressed.gz
[hashtests]$ cd ..
[~]$ ./hashsum ./hashtests
compressed.gz: 35a0ec3087535c23863a774497881ada gzip compressed data, from Unix, [...]
emptyfile: d41d8cd98f00b204e9800998ecf8427e empty
textfile: 536b30bbea5ee48aca04807038e25875 ASCII text
[~]$
\end{verbatim}

\osueguidelinestwo

\end{document}
