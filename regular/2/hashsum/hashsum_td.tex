% TeX source file
% Sysprog WS 2010
% Beispiel 2: Hashsum
% Sebastian Falbesoner
\input{../../template.ltx}

\begin{document}

\osuetitle{2}

\section*{Aufgabenstellung -- hashsum}

Schreiben Sie ein Programm, welches für alle Dateien in einem angegebenen
Verzeichnis jeweils eine Zeile mit dem dazugehörigen MD5-Hashwert und dem
Dateityp ausgibt. Dafür sollen die Standardtools \osueprog{ls} (mit Parameter
\osuearg{-1a}), \osueprog{md5sum}, und \osueprog{file} verwendet werden, indem
diese als Kindprozesse per \osuefunc{fork()}/\osuefunc{exec()} aufgerufen
werden.

\begin{verbatim}
    SYNOPSIS:
        hashes [-i ignorepraefix] <directory>
\end{verbatim}

Mit dem Parameter \osuearg{-i} (``ignore'') werden Dateien, die mit dem
darauffolgenden Muster (nicht case-sensitive!) beginnen, ausgenommen;
so führt ein Aufruf \osuecmd{hashes -i . \~} dazu, dass alle versteckten
Dateien im Home-Verzeichnis nicht behandelt werden.

\subsection*{Anleitung}
Nachdem mit dem Kindprozess \osueprog{ls} der gesamte Inhalt des angegebenen
Verzeichnisses ermittelt und an den Vaterprozess zurückgeben wird, erzeugt man
in einer Schleife für jede der von \osueprog{ls} empfangenen Dateinamen jeweils
einen Kindprozess, der \osueprog{md5sum} mit diesem Namen als Argument ausführt,
gleiches danach für \osueprog{file}. Weiters werden die gewonnenen Informationen
jeweils in einer Zeile (Format: \osueoutform{Dateiname md5summe dateityp})
ausgegeben. Beachten Sie, dass es sich nicht bei allen von \osueprog{ls}
gelisteten Einträgen um Dateien handelt (z.B.\ \osuefilename{.} und
\osuefilename{..}); in dem Fall liefert dann \osueprog{md5sum} einen
nicht-erfolgreichen Return-Code zurück (welcher mittels
\osuefunc{wait()}\osuefunc{waitpid()} ermittelt werden kann) und der Eintrag
soll in diesem Fall ignoriert werden.

\subsection*{Beispielaufruf}

\begin{osuefmtcode}
~ $ \osueinput{mkdir hashtests}
~/hashtests $ \osueinput{cd hashtests}
~/hashtests $ \osueinput{touch emptyfile}
~/hashtests $ \osueinput{echo "das ist das haus vom nikolaus" > textfile}
~/hashtests $ \osueinput{echo "compressme" \textup{|} gzip -cf > compressed.gz}
~ $ \osueinput{cd ..}
~ $ \osueinput{./hashsum ./hashtests}
compressed.gz 35a0ec3087535c23863a774497881ada gzip compressed data, from Unix, [...]
emptyfile d41d8cd98f00b204e9800998ecf8427e empty
textfile 536b30bbea5ee48aca04807038e25875 ASCII text
\end{osuefmtcode}

\osueguidelinestwo

\end{document}
