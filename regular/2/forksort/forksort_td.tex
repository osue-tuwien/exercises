\documentclass{article}
\usepackage[german]{babel}
\usepackage[T1]{fontenc}
\usepackage[utf8]{inputenc}
\usepackage{url}
\usepackage{a4wide}
\parindent0pt
\addtolength{\parskip}{10pt}

\begin{document}

\begin{center}
\begin{Large}
OPERATING SYSTEMS BEISPIEL 2
\end{Large}
\end{center}

\section{Forksort}

\subsection{Aufgabenstellung}
Schreiben Sie ein Programm, das Eingaben sortiert.
\begin{verbatim}
    SYNOPSIS
        forksort
\end{verbatim}

\subsection{Anleitung}
Das Programm soll die Daten von \emph{stdin} lesen. Die erste Zeile muss eine
Zahl größer Null sein, die die Anzahl der zu sortierenden Strings angibt.
Die darauffolgenden Zeilen enthalten die Strings selbst; Sie können die
Strings auf je 62 echte Zeichen begrenzen. (Vergessen Sie nicht dafür eine
Konstante zu definieren!)

\subsection{Arbeitsweise}
Forksort funktioniert wie eine rekursive Merge-Sort-Variante.
\begin{enumerate}
\item Lesen Sie die Anzahl der Strings ein.
\item Ist die Anzahl gleich Eins, dann geben Sie das „sortierte“
  Ergebnis aus. (Es ist einfach der eine String selbst.)
\item Ist die Anzahl größer Eins, dann erstellen Sie vier Unnamed-Pipes (zwei
  je Kind) und forken diese zwei Kindprozesse. Verwenden Sie \emph{dup2(2)} und
  \emph{execlp(3)} um \emph{stdin/stdout} der neuen Prozesse auf eigene Streams
  zu verbiegen und danach Forksort rekursiv aufzurufen.
\item Schreiben Sie je die Hälfte der Strings auf die \emph{stdin} der
  beiden Kinder. Ist die Anzahl der Eingaben ungerade, dann wählen Sie
  ein beliebiges Kind aus, das einen String mehr bekommt. Vergessen Sie nicht
  auch die Anzahl der Strings an das jeweilige Kind zu übermitteln.
\item Lesen Sie nun die Strings der Kinder ein, und geben Sie diese vereinigt
  aus. (Dieser Vorgang nennt sich \emph{Merge}.) Achten Sie darauf, dass die
  beiden Streams schon (für sich alleine) sortiert sind. Sie können also
  solange die Strings des ersten Kindes ausgeben, bis der kleinste String des
  zweiten Kindes kleiner als der kleinste, noch nicht ausgegebene, String des
  ersten Kindes ist. Danach lesen Sie vom zweiten Kind, bis der String des
  ersten Kindes wiederum kleiner ist. Wiederholen Sie diesen Vorgang solange,
  bis alle Strings ausgegeben sind.

  Der Merge-Vorgang muss einen linearen Aufwand haben.
\item Verwenden Sie \emph{wait(2)} oder \emph{waitpid(2)}, um den Exit-Status
  der beiden Kinder zu lesen.
\end{enumerate}

\subsection{Hinweise}
Achten Sie darauf, dass das rekursive Fork nicht zu einer Endlos-Schleife führt.
Dazu sind zwei Regeln zu beachten:
\begin{enumerate}
\item Forken Sie nur, wenn die Anzahl größer eins ist.
\item Die an das jeweilige Kind übergebene Anzahl muss stets kleiner sein als
  die des Vaters.
\end{enumerate}
Beginnen Sie am besten, indem Sie eine Variante schreiben, die nur einen String
sortieren kann. Erweitern Sie nun die Funktion für zwei Strings, indem Sie ein
Fork implementieren und jeweils einen String an jedes Kind schreiben. Diese
können ja schon einen einzelnen String sortieren. Ein einfaches
\emph{strcmp(3)} als Merge-Implementation sollte jetzt noch reichen. Wenn das
funktioniert, können Sie Ihre Merge-Implementierung verallgemeinern, um auch
mehr als zwei Strings sortieren zu können.

Um Fehlermeldungen und Debug-Messages auszugeben, verwenden Sie stets
\emph{stderr}, da die \emph{stdout} in den meisten Fällen verbogen ist.

Zum Testen der Merge-Implementierung können Sie mit \emph{execlp(3)} statt
Forksort auch \emph{sort(1)} aufrufen. Für die endgültige Abgabe ist diese
Vorgehensweise nicht gültig, zum Testen aber durchaus sinnvoll.

\section*{Richtlinien}
Bitte beachten Sie auch die Allgemeinen Hinweise zur Beispielgruppe 2 und die Richtlinien f\"ur die Erstellung von C-Programmen auf der \"Ubungswebsite.
Insbesondere ist es ab dieser Beispielgruppe notwendig, die Dokumentation in Doxygen zu f\"uhren. Es muss zumindest das HTML Output generierbar sein. Bitte dokumentieren Sie ausnahmslos alle Funktionen (auch static-Funktionen, siehe \verb|EXTRACT_STATIC| im Doxygen Cfg-File). Eine kurze Einf\"uhrung haben wir Ihnen auf: \url{http://wiki.vmars.tuwien.ac.at/index.php/Doxygen_Primer} bereitgestellt. Achten Sie weiters darauf, dass nach au{\ss}en hin sichtbare Funktionen (exportierte Funktionen) im Header File beschrieben werden und lokale (static Funktionen) nur im C File. Sie sollten auch Ihre Typen (insb. structs), Konstanten und globale Variablen dokumentieren. 
\end{document}
