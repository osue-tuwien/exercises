\input{../../template.ltx}
\usepackage{amsmath}
\usepackage{multicol}
\begin{document}

\osuetitle{2}

\section*{Assignment -- Closest Pair of Points}
Implement a program which searches for the closest pair of points\footnote{\url{https://en.wikipedia.org/wiki/Closest\_pair\_of\_points\_problem}}
in a set of 2D-points.
\begin{verbatim}
    SYNOPSIS
        cpair
\end{verbatim}

\subsection*{Instructions}
The program takes an array of 2D points as input.
The input is read from \texttt{stdin}.
Each line consists of two floating point numbers,
separated by whitespace,
which represent the $x$ and $y$ coordinates of one point.
The array ends when an EOF (End Of File) is encountered.

Your program must accept any number of points!
Terminate the program with exit status \verb|EXIT_FAILURE|
if an invalid input is encountered.

The program computes the closest pair of points from its input values recursively,
i.e. by calling itself:

\begin{enumerate}
\item If the array consists of only 1 point, then the program exits without generating any output.

\item If the array consists of two points, then write these to \texttt{stdout} and exit.

\item Otherwise the array consists of $n>2$ points.
Calculate $x_m$, the arithmetic mean of the $x$-coordinates of all the points.
Divide the array into two parts,
with one part containing all the points
with an $x$-coordinate less than (or equal to) $x_m$
and the other part containing all the points
with an $x$-coordinate greater than $x_m$.
\footnote{If all $x$-coordinates have the same value then this does not divide the array into two parts of equal length. It is not required to handle this special case. Your program's behavior may be undefined for such inputs (including infinite recursion).} 

\item Using \osuefunc{fork(2)} and \osuefunc{execlp(3)},
recursively execute this program in two child processes,
one for each of the two parts.
Use two unnamed pipes per child
to redirect \osueglvar{stdin} and \osueglvar{stdout}
(see \osuefunc{pipe(2)} and \osuefunc{dup2(3)}).
Write one part to \osueglvar{stdin} of one child
and the other part to \osueglvar{stdin} of the other child.
Read the respective results from each child's \osueglvar{stdout}.
The two child processes must run simultaneously!

\item Use \osuefunc{wait(2)} or \osuefunc{waitpid(2)}
to read the exit status of the children.
Terminate the program with exit status \verb|EXIT_FAILURE|
if the exit status of any of the two child processes is not \verb|EXIT_SUCCESS|.

\item Let $P_1$ be the closest pair of the first part and $P_2$ the closest pair of the second part.

\item For each point in the first part,
calculate its distance to each point in the second part.
Remember the pair with the shortest distance as $P_3$.

\item Compare the distances of $P_1$, $P_2$ and $P_3$
and write the pair with the shortest distance to \texttt{stdout},
with the two points separated by a newline.
Terminate the program with exit status \verb|EXIT_SUCCESS|.
\end{enumerate}

Throughout the program,
it is sufficient to use single precision floating point numbers.

\clearpage
\subsection*{Hints}
\begin{enumerate}
\item You may use the header \osueglvar{math.h} for the distance calculations.
Add \verb|-lm| to the linker options to use these functions.
\item Create a structure to store the coordinates of a point.
\item In order to avoid endless recursion\footnote{\url{http://en.wikipedia.org/wiki/Fork\_bomb}},
fork only if the input number is greater than 1.
\item To output error messages and debug messages, always use
\osueglvar{stderr} because \osueglvar{stdout} is redirected in most cases.
%\item Think about some edge cases and handle it accordingly. E.g.: what if all points have the same x value $\rightarrow$ take the median instead of the mean to spilt the array.
\end{enumerate}

\subsection*{Examples}
\begin{verbatim}
$ cat 1.txt
4.0 4.0
-1.0 1.0
1.0 -1.0
-4.0 -4.0
$ ./cpair < 1.txt
-1.000000 1.000000
1.000000 -1.000000
\end{verbatim}

Small deviations of the resulting values,
which are a consequence of the limited precision of floating point numbers,
are not relevant and not considered to be an error.

\input{./../BonusExercise.tex}

\osueguidelinestwo

\end{document}
