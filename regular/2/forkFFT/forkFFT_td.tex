\input{../../template.ltx}
\usepackage{amsmath}
\begin{document}

\osuetitle{2}

\section*{ Assignment -- Fork Fourier Transform}
Implement the Cooley-Tukey Fast Fourier Transform \footnote{https://en.wikipedia.org/wiki/Cooley-Tukey\_FFT\_algorithm} algorithm.
\begin{verbatim}
    SYNOPSIS
        forkFFT
\end{verbatim}

\subsection*{Instructions}
The input of the program are float values, which should be read from \osueglvar{stdin}. Each value is represented in a new line.
\subsection*{Examples}
\begin{verbatim}
$ ./forkFFT
1.0
1.0
\end{verbatim}
Output: 
\begin{verbatim}
	2.0	0.0*i
	0.0	0.0*i
\end{verbatim}
\subsection*{Approach to work}
ForkFFT computes recursively the Fourier Transform from its input values. Recursively means that the program calls itself. 
\begin{enumerate}
\item If the array consists of only 1 number, write that number to \osueglvar{stdout} and exit.
\item Otherwise the array consists of $n$ numbers.
Split it into two parts with $n/2$ numbers each,
such that one part contains all the numbers at even indices
and the second part contains all the numbers at odd indices:
\[
P_E=\{A[0],\;A[2],\;A[4],\;\dots\}
\]
\[
P_O=\{A[1],\;A[3],\;A[5],\;\dots\}
\]
\item Using \osuefunc{fork(2)} and \osuefunc{execlp(3)},
recursively execute this program two times for each of the two parts.
Let $R_E$ be the result of the even part and $R_O$ be the result of the odd part.
The result $R$ of the Fourier Transform of the entire array can now be computed
by applying the ``butterfly'' operation to the two results:
\[
\forall k\;|\;0\leq k<n/2\;:\quad R[k]=R_E[k]+e^{-\frac{2\pi i}{n}\cdot k}\cdot R_O[k]
\quad\text{and}\quad R[k+n/2]=R_E[k]-e^{-\frac{2\pi i}{n}\cdot k}\cdot R_O[k]
\]
which is identical to:
\[
\forall k\;|\;0\leq k<n/2\;:\quad
\renewcommand{\arraystretch}{1.8}
\begin{array}{r@{\;=\;}l}
 R[k]     & R_E[k]+\left(\text{cos}(-\frac{2\pi}{n}\cdot k)+i\cdot\text{sin}(-\frac{2\pi}{n}\cdot k)\right)\cdot R_O[k]\\
R[k+n/2] & R_E[k]-\left(\text{cos}(-\frac{2\pi}{n}\cdot k)+i\cdot\text{sin}(-\frac{2\pi}{n}\cdot k)\right)\cdot R_O[k]
\end{array}
\]
The resulting array is printed to \texttt{stdout}, one value per line.
Since the resulting numbers are complex, the program prints two values per line:
the real and the imaginary part of each value.

\item For the even child (do the same for the odd child) create first two unnamed pipes and create with \osuefunc{fork(2)} a child. Use now \osuefunc{dup2(2)} to redirect \osueglvar{stdin}/\osueglvar{stdout} of the child. In the child process call now forkFFT recursively.
\item Now write the even and odd parts of the array to \osueglvar{stdin} of the children created before. Use the previously specified input format.
\item Read now the output of the two children and calculate the butterfly. 
\item Use \osuefunc{wait(2)} or \osuefunc{waitpid(2)} to read the exit status of the children.

\end{enumerate}

\subsection*{Hints}
Use the libraries math.h and complex.h, which needs an additional linker option $-lm$.
Pay attention to correct termination conditions of forkFFT to avoid endless recursions \footnote{http://en.wikipedia.org/wiki/Fork\_bomb}.
There are two rules to consider:
\begin{enumerate}
	\item Because the Cooley-Tukey algorithm divides the input by 2, the number of inputs is limited to a power-of-two. If the input is not a power of two your program should round up to the next power of two.
\item Fork only if the input number is greater than 1.
\end{enumerate}

To output error messages and debug messages, always use
\osueglvar{stderr} because \osueglvar{stdout} is redirected in most cases.

\osueguidelinestwo

\end{document}
