\input{../../template.ltx}
\usepackage{amsmath}
\usepackage{multicol}
\begin{document}

\osuetitle{2}

\section*{Assignment -- Integer Multiplication}
Implement an algorithm for the efficient multiplication of large integers.
\begin{verbatim}
    SYNOPSIS
        intmul
\end{verbatim}

\subsection*{Instructions}
The program takes two hexadecimal integers $A$ and $B$ with an equal number of digits as input, multiplies them and prints the result.
The input is read from \texttt{stdin} and consists of two lines:
the first line is the integer $A$ and the second line is the integer $B$.

Your program must accept any number of digits.
Terminate the program with exit status \verb|EXIT_ERROR|
if an invalid input is encountered
or if the two integers do not have equal length.

The multiplication of the input values is calculated recursively,
i.e. by the program calling itself:
\begin{enumerate}
\item If $A$ and $B$ both consist of only 1 hexadecimal digit, then multiply them,
write the result to \texttt{stdout} and exit.
\item Otherwise $A$ and $B$ both consist of $n>1$ digits. Split them both into two parts each,
with each part consisting of $n/2$ digits:
\renewcommand{\arraystretch}{1.3}
\[
A:\quad\quad\begin{array}{|c|c|}\hline \quad A_h \quad & \quad A_l \quad \\\hline\end{array}\quad\quad A=A_h\cdot 16^{n/2}+A_l
\]
\[
B:\quad\quad\begin{array}{|c|c|}\hline \quad B_h \quad & \quad B_l \quad \\\hline\end{array}\quad\quad B=B_h\cdot 16^{n/2}+B_l
\]
Terminate the program with exit status \verb|EXIT_ERROR|
if the number of digits is not even.

\item Using \osuefunc{fork(2)} and \osuefunc{execlp(3)},
recursively execute this program in four child processes,
one for each of the multiplications $A_h\cdot B_h$, $A_h\cdot B_l$, $A_l\cdot B_h$ and  $A_l\cdot B_l$.
Use two unnamed pipes per child
to redirect \osueglvar{stdin} and \osueglvar{stdout}
(see \osuefunc{pipe(2)} and \osuefunc{dup2(3)}).
Write the two values to be multiplied to \osueglvar{stdin} of each child.
Read the respective result from each child's \osueglvar{stdout}.
The four child processes must run simultaneously!

\item Use \osuefunc{wait(2)} or \osuefunc{waitpid(2)}
to read the exit status of the children.
Terminate the program with exit status \verb|EXIT_ERROR|
if the exit status of any of the two child processes is not \verb|EXIT_SUCCESS|.

\item The result of the multiplication $A\cdot B$ can now be calculated as follows:
\[
A\cdot B=A_h\cdot B_h\cdot 16^n + A_h\cdot B_l\cdot 16^{n/2} + A_l\cdot B_h\cdot 16^{n/2} + A_l\cdot B_l
\]
Find a clever way to write the result of this operation to \texttt{stdout},
even if the numbers are too large for the C data types.
Remember that your program must deal with integers of any size!

\end{enumerate}

\subsection*{Hints}

\begin{itemize}
\item Think of a way to add the four intermediate results together
one digit at a time while keeping track of the carry.
\item In order to avoid endless recursion\footnote{\url{http://en.wikipedia.org/wiki/Fork\_bomb}},
fork only if the input number is greater than 1.
\item To output error messages and debug messages, always use
\osueglvar{stderr} because \osueglvar{stdout} is redirected in most cases.
\end{itemize}

\subsection*{Examples}
\begin{multicols}{3}
\begin{verbatim}
$ cat 1.txt
3
3
$ ./intmul < 1.txt
9
\end{verbatim}

\begin{verbatim}
$ cat 2.txt
1A
B3
$ ./intmul < 2.txt
122e
\end{verbatim}

\begin{verbatim}
$ cat 3.txt
13A5D87E85412E5F
7812C53B014D5DF8
$ ./intmul < 3.txt
09372e47ae47c3f68e45d1a816906f08
\end{verbatim}
\end{multicols}

\osueguidelinestwo

\end{document}
