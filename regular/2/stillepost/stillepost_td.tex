% TeX source file
% Sysprog WS 2005
% Beispiel 3: stillepost
% Dietmar Schabus (e0147322@student.tuwien.ac.at)
\input{../../template.ltx}

\begin{document}

\osuetitle{2}

\section*{Aufgabenstellung -- stillepost}

Gründen Sie eine Prozessfamilie bestehend aus einem Elternprozess und
$n$ Kindprozessen (nicht Kind und Enkelkind, $n\ge2$). Spielen Sie dann mit
der Familie ein Stille-Post-Spiel.

Das von Ihnen zu entwickelnde Programm \osueprog{stillepost} soll sich
folgendermaßen verhalten:

\begin{itemize}
\item Der Benutzer des Programms überlegt sich einen beliebigen
String.
\item Er ruft das Programm (also den Elternprozess) mit dem String als
Parameter auf. Beispiel: \osuecmd{stillepost 25 \char`"OSUE macht
Spass.\char`"}
\item Der Elternprozess sagt den String an sein erstes Kind weiter,
jedoch nicht ohne eine kleine Veränderung einzubauen -- wie beim
Stille-Post-Spiel üblich.
\item Das erste Kind erhält diesen String, verändert ihn wieder
geringfügig und gibt ihn an das zweite Kind weiter.
\item Das zweite Kind verfährt ebenso wie das erste, nur gibt es den
von ihm veränderten String nicht an einen anderen Prozess weiter,
sondern sagt ihn laut (gibt ihn auf \osueprog{stdout} aus).
\end{itemize}

\subsection*{Anleitung}

Das Programm soll folgende Synopsis befolgen:
\begin{verbatim}
    SYNOPSIS:
        stillepost [-v] <n> <string>
\end{verbatim}

Beachten Sie: Wenn Sie einen String übergeben wollen, der Leerzeichen
enthält, müssen Sie ihn in Hochkommata einschließen, damit er als
nur ein Argument behandelt wird: \osuearg{\char`"String mit
Leerzeichen\char`"}

Der Elternprozess soll seine Kinder mittels \osuefunc{fork(2)}
erzeugen, die Kommunikation zwischen den Prozessen soll über Unnamed
Pipes erfolgen, welche mit \osuefunc{pipe(2)} erzeugt werden.

Die Veränderungen eines jeden Mitspielers (Prozesses) sollen so
vorgenommen werden, dass mittels \osuefunc{rand(3)} eine zufällige
Position innerhalb des Strings gewählt, und dann das Zeichen an
dieser Stelle durch ein anderes, ebenfalls zufälliges Zeichen aus
\osueregex{[a-zA-Z]} ersetzt wird.

Stellen Sie mittels \osuefunc{srand(3)} sicher, dass alle Prozesse eine
unterschiedliche Random-Seed verwenden, damit nicht jeder Prozess die
selben Änderungen vornimmt.

Die Option \osuearg{-v} schließlich soll es ermöglichen, das Spiel aus
der Sicht eines allwissenden Betrachters (bzw. jemand mit ausgezeichnetem Geh{\"o}hr)
zu verfolgen. Wird die Option
angegeben, gibt jeder Prozess den String zuerst unverändert und dann
in der veränderten Form (wie er ihn weitergeben wird) auf
\osueglvar{stdout} aus.

Führen Sie eine sinnvolle Beschränkung für die Länge des Strings
ein.

\subsection*{Testen}

Testen Sie ihr Programm vor allem mit der Option \osuearg{-v}, da dann
deutlich sichtbarer ist, was jeder einzelne Prozess tut.

Beispielrunde:

\begin{osuefmtcode}
$ \osueinput{./stillepost -v 2 "OSUE macht Spass."}
Parent: erhalten: OSUE macht Spass.
Parent: weiter  : OSUE macht Xpass.
child1 : erhalten: OSUE macht Xpass.
child1 : weiter  : OSkE macht Xpass.
child2 : erhalten: OSkE macht Xpass.
child2 : weiter  : OSkE macAt Xpass.
child2 : Ende    : OSkE macAt Xpass.
\end{osuefmtcode}

Die Ausgabe ihres Programms muss nicht genau wie in dem Beispiel
formatiert sein. Natürlich wird der String umso unkenntlicher, je
kürzer er ist.

\osueguidelinestwo

\end{document}
