% TeX source file
% Sysprog SS 2005
% Beispiel 3: encr
% Christian Steiner
\input{../../template.ltx}

\begin{document}

\osuetitle{2}

\section*{Aufgabenstellung -- encr}

Implementieren Sie ein Programm, welches kontinuierlich Strings
(Passwörter) von \osueglvar{stdin} einliest und diese von einem Kindprozess
verschlüsseln lässt.

\begin{verbatim}
    SYNOPSIS:
        encr
\end{verbatim}

Jedes Mal, wenn der Vaterprozess eine Zeile eingelesen hat, wird ein
eigener Kindprozess erzeugt, der das Passwort (exklusive des
abschließenden Newline-Zeichens) mittels \osuefunc{crypt(3)}
verschlüsselt. Danach soll der Kindprozess mittels \osuefunc{sleep(3)}
eine längere (pseudozufällige) Verarbeitungsdauer simulieren, um das
nichtblockierende Einlesen des Vaterprozesses zu demonstrieren, und
schließlich das verschlüsselte Passwort auf \osueglvar{stdout} ausgeben.

\subsection*{Anleitung}

Lesen Sie in einer Schleife zeilenweise Strings von \osueglvar{stdin} ein
und entfernen Sie das abschließende Newline (\verb+\n+). Erzeugen Sie
mittels \osuefunc{fork(2)} einen Kindprozess und verschlüsseln Sie in
diesem das Passwort; der Vaterprozess liest zu diesem Zeitpunkt
bereits wieder von \osuefunc{stdin} ein. Beachten Sie, dass in dieser
Situation mehrere Kindprozesse asynchron und in nicht vorhersehbarer
Reihenfolge terminieren können!

Der Exit-Status jedes Kindes muss so früh wie möglich abgeholt werden.
Behandeln Sie das Signal \osueconst{SIGCHLD}, um das zu gewährleisten.
Achten Sie darauf, dass \osuefunc{fgets(3)} den Fehlercode \osueconst{EINTR}
liefert, wenn es durch ein Signal unterbrochen wird. In diesem Fall
muss \osuefunc{fgets(3)} erneut aufgerufen werden.

\subsection*{Testen}

Testen Sie ihr Programm z.B.\ wie folgt

\begin{osuefmtcode}
$ \osueinput{./encr}
\osueinput{jamesbond}
\osueinput{micky}
encr: jamesbond -> aCqHsO4OYLnq.
encr: micky -> aCSHDy093k9FQ
\osueinput{turing07}
encr: turing07 -> aCJvjJDrpvXWE
\osueinput{asdfqwer}
encr: asdfqwer -> aCZpdSZB6ohMQ
\end{osuefmtcode}

Die Ausgabe ihres Programmes muss nicht genau wie in diesem Beispiel
formatiert sein. Die Reihenfolge von Input und Output wird vermutlich
anders sein, ebenso wie das verschlüsselte Wort je nach gewähltem Salt
variiert.

Beachten Sie beim Testen auch den Fall, in dem der Vaterprozess durch
ein Signal terminiert wird, während noch Kindprozesse aktiv sind. Auch
in diesem Fall muss der Vaterprozess auf das Terminieren seiner Kinder
warten.

\osueguidelinestwo

\end{document}
