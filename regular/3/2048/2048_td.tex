\input{../../template.ltx}

\begin{document}

\osuetitle{3}

\section*{Aufgabenstellung -- 2048}

In dieser Aufgabe werden Sie eine Terminal Version des Spiels "`2048"' (auch
bekannt als "`1024"' und "`Threes"') implementieren.

Das Spielfeld besteht aus 16 Feldern, die in einem 4x4 Quadrat angeordnet sind.
Jedes davon kann entweder leer sein oder einen Wert
aus der Menge $\{ 2^i | i \in \mathbb{N}^+\}$ enthalten.

Benachbarte Felder, welche den selben Wert beinhalten, können kombiniert werden
und verschmelzen zu einem Feld mit dem Wert der Summe beider vorherigen Felder.
Zum Beispiel ergibt eine Kombination von 2 Feldern mit Wert \verb|32| ein Feld
mit Wert \verb|64|.

Ziel des Spiels ist, Felder geschickt zu kombinieren, um Felder mit möglichst
hohen Werten entstehen zu lassen.  Ein Spiel ist gewonnen sobald ein Feld mit
Wert \verb|2048| entsteht (bzw.\ der Wert, mit dem der Server als
Optionsargument gestartet wurde, siehe Abschnitt~\ref{sec:server}).  Falls das
Spielfeld voll ist und keine weiteren Kombinationsmöglichkeiten bestehen, hat
der Spieler verloren.

In einer Spielrunde werden folgende Schritte durchlaufen:

\begin{enumerate}
\item Ein freies Feld wird zufällig ausgewählt und mit einem neuen Wert
      befüllt.
      % Dabei soll ein Wert von $2^i, i > 0$ mit einer Wahrscheinlichkeit
      % von $2^{-i}$ erzeugt werden.
      Dabei soll der Wert \verb|2| mit einer Wahrscheinlichkeit von 0.75 oder
      der Wert \verb|4| mit einer Wahrscheinlichkeit von 0.25 erzeugt werden.
\item Der Spieler gibt eine Richtung an, in welche alle Felder mit Werten
      auf dem Spielfeld verschoben werden.
      \label{stepinput}
\item Daraufhin wird der angegebene Zug ausgeführt. Im folgenden steht $r$ für
      die vom Spieler angegebene Richtung. Angefangen auf der Seite $r$
      wird nun für jedes Feld $f$ folgende Logik angewandt: \label{stepturn}
      \begin{enumerate}
      \item Solange das Nachbarfeld von $f$ in Richtung $r$ leer ist,
            verschiebe $f$ dorthin.
      \item Falls das Nachbarfeld von $f$ in Richtung $r$ den selben Wert hat
            wie $f$, entferne $f$ und verdopple den Wert vom Nachbarfeld.
      \end{enumerate}
\item Falls das Spielfeld durch Schritt~\ref{stepturn} unverändert geblieben
      ist, setzen Sie mit Schritt~\ref{stepinput} fort; andernfalls beginnt
      eine neue Spielrunde.
\end{enumerate}

Bei Fragen zu den Spielregeln und für weitere Inspiration, orientieren Sie sich
am einfachsten an\\
\url{http://gabrielecirulli.github.io/2048/}.


\subsection*{Anleitung}

Das Spiel soll als Client/Server Programm realisiert werden, wobei ein Server
beliebig viele Clients (das heißt auch beliebig viele gleichzeitige Spiele)
bedient. Kommunikation zwischen Clients und Server soll per Shared Memory
erfolgen.

\subsubsection*{Server}
\label{sec:server}

\begin{verbatim}
USAGE: 2048-server [-p power_of_two]
    -p:     Play until 2^power_of_two is reached (default: 11)
\end{verbatim}

Über die Option \verb|-p| kann angegeben werden, wann das Spiel als gewonnen
gilt, also bei welcher Zweierpotenz, die als Wert auf einem Feld erreicht wird.
Per default beträgt dieser Wert $2^{11} = 2048$.

Spielelogik findet ausnahmslos am Server statt, während Clients lediglich für
Darstellung und Benutzereingaben verantwortlich sind. Eine Nachricht zum Client
soll also das aktuelle Spielfeld und Status Codes (z.B.\ \verb|ST_WON|,
\verb|ST_LOST|, \verb|ST_NOSUCHGAME|) enthalten, während eine Nachricht zum
Server nur Befehle (z.B.\ \verb|CMD_LEFT|, \verb|CMD_QUIT|) enthält.

Der Server soll eine unbegrenzte Anzahl an gleichzeitigen Spielen verwalten
können.  Sobald ein Client ein Request für ein neues Spiel sendet, sollen lokal
benötigte Ressourcen angelegt werden. Bei Beendigung eines Spiels müssen Sie
diese natürlich wieder freigeben.

Clients kommunizieren mit dem Server über ein Shared Memory Segment. Zugriffe
darauf sollen mit Semaphoren synchronisiert werden. Stellen Sie sicher, dass
nach einem Request von Client $C_i$ die Antwort unbedingt nur $C_i$ (und nicht
einen anderen Client $C_j, j \neq i$) erreicht.

Jedem Spiel wird vom Server eine ID zugeteilt, und Clients können sich entweder
zu bestehenden Spielen verbinden oder ein neues erstellen. Insbesondere werden
Spiele am Server nur gelöscht falls sie entweder vom Client explizit beendet
werden oder verloren sind. Ein Client soll sich also vom Spiel trennen und
später wieder zum selben Spiel verbinden können. Sie dürfen annehmen, dass sich
zu einem Spiel nie mehr als ein Client gleichzeitig verbindet.

Falls der Server ein \verb|SIGTERM| oder \verb|SIGINT| Signal erhält, sollen
alle angelegten Ressourcen (lokal angelegter Speicher, Shared Memory,
Semaphoren, etc) korrekt freigegeben und terminiert werden.

\subsubsection*{Client}

\begin{verbatim}
USAGE: 2048-client [-n | -i <id>]
    -n:     Start a new game
    -i:     Connect to existing game with the given id
\end{verbatim}

Der Client ist für Darstellung der Spielfelds und Behandlung von
Benutzereingaben zuständig.

Die Gestaltung der Oberfläche ist nicht weiter spezifiziert, außer dass
mindestens das Spielfeld, die ID des aktuellen Spiels, und eine kurze Anleitung
mit den möglichen Aktionen angezeigt werden müssen.

Mögliche Aktionen sind:

\begin{itemize}
\item \emph{Links/Rechts/Oben/Unten}: Angabe der Richtung eines Spielzugs.
      Taste \verb|a| steht für links, \verb|d| für rechts, \verb|w| für oben,
      \verb|s| für unten.
\item \emph{Spiel Löschen}: Das Spiel wird aufgegeben, Ressourcen werden am
      Server gelöscht, der Client terminiert.
\item \emph{Verbindung Trennen}: Der Client terminiert ohne Spielressourcen am
      Server zu löschen.
\end{itemize}

Der Client soll zusätzlich ordnungsgemäß terminieren sobald das Spiel gewonnen
oder verloren ist, oder der Server (d.h.\ das Shared Memory oder die Semaphore)
nicht mehr erreichbar ist.

\osueguidelinesthree

\end{document}
