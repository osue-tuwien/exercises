\input{../../template.ltx}

\begin{document}

\osuetitle{3}

\section*{Aufgabenstellung}

Schreiben sie eine Store-Tool Anwendung. Diese Anwendung besteht aus zwei Prozessen.

\begin{verbatim}
    SYNOPSIS
        storeserver
        storetool  <-s key data | -g key | -r key | -q>
\end{verbatim}

Die beiden Prozesse kommunizieren über ein {\em Shared-Memory}. Das Store-Tool schickt Kommandos an den Server. Danach wartet es bis der Server das Kommando abgearbeitet hat, überprüft das Ergebnis und terminiert.
Gespeichert wird ein String, den sie auf 64 Zeichen inklusive terminierendes '$\backslash$0' (Konstante definieren!) begrenzen können. Dieser String wird mit einem Schlüssel (\osuearg{key}) assoziiert und soll ein positiver, ganzzahliger Wert sein.

\subsection*{Kommandos}
\begin{center}
\begin{tabular}{llp{10cm}}
\hline
store & \osuearg{-s key data} & Der Server soll die Daten \osuearg{'data'} assoziiert mit \osuearg{key} speichern. Existiert \osuearg{key} bereits dann meldet der Server einen Fehler der vom Store-Tool abfangen soll. 

Verwenden sie zum Speichern eine einfach verkettete Liste. \\ 
\hline
get & \osuearg{-g key} & Das Store-Tool fordert vom Server die mit \osuearg{key} assoziierten Daten an. Wenn keine Daten gespeichert sind soll der Server einen Fehler an das Store-Tool melden. Tritt kein Fehler auf, dann soll das Store-Tool die Daten auf \osueglvar{stdout} ausgeben. \\
\hline
remove & \osuearg{-r key} & Der Server soll einen Wert aus der Liste löschen. Existiert \osuearg{key} nicht, dann soll das Store-Tool einen Fehler ausgeben \\
\hline
quit & \osuearg{-q} & Erhält der Server dieses Kommando, dann soll er sich beenden. \\
\hline
\end{tabular}
\end{center}
 
\subsection*{Hinweis}
Nicht kritische Fehlermeldungen (jene die sich auf die Liste beziehen) sollen vom Store-Tool auf \osueglvar{stderr} ausgegeben werden. Es ist darauf zu achten, daß mehrere Store-Tools gleichzeitig gestartet werden können. Das Zweite muss in diesem Fall so lange warten bis das Erste das Ergebnis vom Server gelesen hat.

Bei dem Kommando \emph{quit} dürfen schon wartende Store-Tools mit einem Fehler terminieren. Nicht aber der Server und jenes Store-Tool das das Kommando sendet.

\subsection*{Anleitung}
Schreiben sie zwei Programme, die die Prozesse mittels einer Client/Server-Struktur realisieren. Achten sie auf eine saubere Terminierung.

\osueguidelinesthree

\end{document}
