\input{../../template.ltx}

\begin{document}

\osuetitle{3}

\section*{Aufgabenstellung -- storetool}

Schreiben sie eine Store-Tool-Anwendung. Diese Anwendung besteht aus zwei
Prozessen.

\begin{verbatim}
    SYNOPSIS
        storeserver
        storetool  <-s key data | -g key | -r key | -q>
\end{verbatim}

Die beiden Prozesse kommunizieren über ein Shared Memory. Das Store-Tool
schickt Kommandos an den Server. Danach wartet es, bis der Server das Kommando
abgearbeitet hat, überprüft das Ergebnis und terminiert.
Gespeichert wird ein String, den sie auf 64 Zeichen inklusive terminierendem
\verb_'\0'_ (Konstante definieren!) begrenzen können. Dieser String wird mit
einem Schlüssel (\osuearg{key}) assoziiert und soll ein positiver, ganzzahliger
Wert sein.

\subsection*{Kommandos}
\begin{center}
\begin{tabular}{@{}llp{10cm}@{}}
\toprule
store  & \osuearg{-s key data} & Der Server soll die Daten (\osuearg{data})
                                 assoziiert mit \osuearg{key} speichern.
                                 Existiert \osuearg{key} bereits, meldet der
                                 Server einen Fehler, der vom Store-Tool
                                 abgefangen werden soll.

                                 Verwenden Sie zum Speichern eine einfach
                                 verkettete Liste. \\
\midrule
get    & \osuearg{-g key}      & Das Store-Tool fordert vom Server die mit
                                 \osuearg{key} assoziierten Daten an. Wenn keine
                                 Daten gespeichert sind, soll der Server einen
                                 Fehler an das Store-Tool melden. Tritt kein
                                 Fehler auf, soll das Store-Tool die Daten auf
                                 \osueglvar{stdout} ausgeben. \\
\midrule
remove & \osuearg{-r key}      & Der Server soll einen Wert aus der Liste
                                 löschen. Existiert \osuearg{key} nicht, soll
                                 das Store-Tool einen Fehler ausgeben. \\
\midrule
quit   & \osuearg{-q}          & Erhält der Server dieses Kommando, soll er sich
                                 beenden. \\
\bottomrule
\end{tabular}
\end{center}

\subsection*{Hinweis}

Nicht-kritische Fehlermeldungen (jene, die sich auf die Liste beziehen) sollen
vom Store-Tool auf \osueglvar{stderr} ausgegeben werden. Es ist darauf zu
achten, dass mehrere Store-Tools gleichzeitig gestartet werden können. Das
zweite muss in diesem Fall so lange warten, bis das erste das Ergebnis vom
Server gelesen hat.

Bei dem Kommando \osueinput{quit} dürfen schon wartende Store-Tools mit einem
Fehler terminieren, nicht aber der Server und jenes Store-Tool, das das Kommando
sendet.

\subsection*{Anleitung}

Schreiben Sie zwei Programme, die die Prozesse mittels einer
Client/Server-Struktur realisieren. Achten Sie auf eine saubere Terminierung.

\osueguidelinesthree

\end{document}
