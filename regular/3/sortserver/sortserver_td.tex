\input{../../template.ltx}

\begin{document}

\osuetitle{3}

\section*{Aufgabenstellung}

Schreiben Sie einen Sortier-Server und einen dazu passenden Client.

\begin{verbatim}
    SYNOPSIS
        sortserv
        sortclient
\end{verbatim}

Der Server soll aus einem Shared Memory einzelne Zahlen vom Typ
\osuekeyword{long} einlesen und diese anschließend sortieren. Die sortierte
Zahlenfolge soll dann auf die gleiche Art und Weise zurück an den Client
geschickt werden.

Der Client liest Zahlen von \osueglvar{stdin} und schreibt Sie in den Shared
Memory. Nach der Eingabe von \osueconst{EOF} (\osuekeystroke{Ctrl+D}) beginnt
der Client die sortierte Zahlenfolge aus dem Shared Memory zu lesen und gibt die
vom Server sortierten Werte auf \osueglvar{stdout} aus.

\subsection*{Hinweis}
Es sollen beliebig lange Zahlenfolgen erlaubt sein, die Größe des Shared Memory
hingegen soll vor der Laufzeit auf einen konstanten Wert festgelegt werden.
Versuchen Sie die Größe des Shared Memory zu minimieren und auf einige wenige
Bytes zu beschränken.

\subsection*{Anleitung}
Schreiben Sie zwei Programme, die die Prozesse mittels einer
Client/Server-Struktur realisieren. Achten Sie auf eine saubere Terminierung,
nachdem alle Zahlen ausgegeben sind. Verwenden Sie zum Sortieren die Funktion
\osuefunc{qsort(3)}.

\osueguidelinesthree

\end{document}
