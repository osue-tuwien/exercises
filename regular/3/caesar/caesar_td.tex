% TeX source file
% Sysprog WS 2005
% Beispiel 2: caesar
% Dietmar Schabus (e0147322@student.tuwien.ac.at)
\input{../../template.ltx}

\begin{document}

\osuetitle{3}

\section*{Aufgabenstellung -- caesar}

\begin{verbatim}
    SYNOPSIS:
        readin <filename>
        caesar <key>
\end{verbatim}

Implementieren Sie zwei Programme \osueprog{readin} und \osueprog{caesar}, die
einen einfachen (und leicht zu knackenden!) Verschlüsselungsalgorithmus
anwenden.

Das Programm \osueprog{readin} liest zeilenweise aus einer als Argument
übergebenen Textdatei und schreibt jede Zeile nacheinander auf folgende Weise
modifiziert in ein Shared Memory:

\begin{itemize}
\item Alle Großbuchstaben (alle Zeichen in \osueregex{[A-Z]}) werden
unverändert beibehalten.
\item Alle Kleinbuchstaben (alle Zeichen in \osueregex{[a-z]}) werden in
Großbuchstaben konvertiert.
\item Alle anderen Zeichen fallen weg.
\end{itemize}

Das Programm \osueprog{caesar} erwartet als Argument eine Zahl zwischen 1
und 25. Es liest jede Zeile aus dem Shared Memory und ersetzt jeden
Buchstaben durch den im Alphabet um \osuearg{key} weiter rechts stehenden
Buchstaben (wobei sich rechts neben dem Z wieder das A befinden
soll). Bei einem \osuearg{key} von 3 wird also z.B.\ A zu D, B zu E, C zu
F, W zu Z, X zu A und Y zu B. Die so verschlüsselte Zeile wird dann
auf \osueglvar{stdout} ausgegeben, und die nächste Zeile wird gelesen.

\osueprog{caesar} soll die nächste Zeile erst dann lesen, wenn \osueprog{readin}
sie vollständig ins Shared Memory geschrieben hat, und \osueprog{readin} soll
erst dann die nächste Zeile ins Shared Memory schreiben, wenn die vorherige
schon von \osueprog{caesar} gelesen wurde.

Sobald \osueprog{readin} die letzte Zeile der Eingabedatei verarbeitet hat,
schreibt es einen frei wählbaren String in das Shared Memory, der
\osueprog{caesar} anzeigt, dass alle Arbeit getan ist. Daraufhin sollen beide
Programme terminieren.

\subsection*{Anleitung}

Folgende vereinfachenden Einschränkungen gelten:

\begin{itemize}
\item In der Eingabedatei stehen nur \textsc{ascii}-Zeichen. Umlaute und
dergleichen müssen also nicht behandelt werden.
\item Keine Zeile ist länger als 1022 Zeichen (inklusive Newline).
\end{itemize}

\subsection*{Testen}

Testen Sie Ihr Programm mit mehreren Eingabedateien.

Erstellen Sie z.B.\ eine Testdatei \osuefilename{t1} mit folgendem Inhalt:
\begin{verbatim}
Ich habe das dritte Sysprog-Beispiel,
das gar nicht schwer war,
schon fast fertig.
\end{verbatim}
Rufen Sie dann die beiden Programme auf (Reihenfolge je nach Ihrer
Implementierung):
\begin{verbatim}
  readin t1 &
  caesar 15 > t2
\end{verbatim}
Dann haben Sie in \osuefilename{t2} die verschlüsselte Version von
\osuefilename{t1}.

Mit
\begin{verbatim}
  readin t2 &
  caesar 11
\end{verbatim}
können Sie diese wieder entschlüsseln, und sollten folgende Ausgabe
erhalten:
\begin{verbatim}
ICHHABEDASZWEITESYSPROGBEISPIEL
DASGARNICHTSCHWERWAR
SCHONFASTFERTIG
\end{verbatim}

\osueguidelinesthree

\end{document}
