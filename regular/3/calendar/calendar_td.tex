\documentclass{article}
\usepackage[german]{babel}
\usepackage{a4wide}
\parindent0pt

\usepackage[utf8x]{inputenc}
%\usepackage[latin1]{inputenc}
\usepackage[T1]{fontenc}
%\chapterstyle{veelo}
%\usepackage{ucs}
\usepackage{amsmath}
\usepackage{amsfonts}
\usepackage{amssymb}
\usepackage{amsthm}

\setlength\parindent{0pt}
\setlength{\parskip}{1.5ex}

\begin{document}

\begin{center}
\LARGE OPERATING SYSTEMS UE BEISPIEL 3
\end{center}

\section*{Aufgabenstellung}

Ganz knapp vor der Prüfung für Betriebssysteme, auf die Sie sich seit 
Monaten freuen, erfahren Sie, dass all Ihre Freunde dachten, die Prüfung fände 
drei Tage später statt. Sofort evaluieren Sie deren Fehler und merken, das Problem
liegt am fehlenden Zeitmanagement Ihrer Kollegen. Hilfsbereit, wie Sie sind, wollen 
Sie sie unterstützen und schreiben zu diesem Behufe eine simple Kalenderapplikation.

\begin{verbatim}
    SYNOPSIS
        calserver [-v]
        calclient [-r idx] [-d offset] [entry]
\end{verbatim}

\subsection*{Details}

Server und Client sollen über Shared Memory miteinander 
kommunizieren. Der Server verwaltet hierbei die Kalenderdaten, 
mit der Option \emph{-v} gibt er Auskunft über die Aktionen, 
die er dabei durchführt.\\
Der Client fungiert gleichzeitig als Eingabe- und als Ausgabeapplikation. 
Soll ein neuer Eintrag eingefügt werden, so wird \emph{calclient} mit 
\emph{entry} als Text für diesen Eintrag aufgerufen. Ohne Angabe von \emph{entry} 
sollen die Tageseinträge angezeigt werden.
Mit dem Argument \emph{offset} wird der zu bearbeitende Tag relativ zum heutigen
angegeben, beispielsweise würde \emph{-d 1} den morgigen Tag auswählen. Mit 
\emph{-r idx} wird am ausgewählten Tag der Eintrag an Stelle \emph{idx} entfernt 
oder, falls \emph{entry} mitangegeben wird, damit überschrieben.

\subsection*{Beispiele}

\begin{verbatim} calclient 
Output: 
1: Prf BSys
2: Abgabe Beispiel 1
\end{verbatim}
Zeigt heutige Termine an.

\begin{verbatim} calclient -d 3 "VO BSys" 
\end{verbatim}
Legt den Eintrag \emph{VO BSys} in drei Tagen an.

\begin{verbatim} calclient -r 2 "Prf BSys" 
\end{verbatim}
Ersetzt den zweiten Eintrag am heutigen Tag durch \emph{Prf BSys}.

\subsection*{Hinweise}
Überlegen Sie sich eine geeignete Datenstruktur zum Speichern der Einträge. Als 
Schlüssel eignen sich u.a. UNIX-Timestamps (siehe dazu \emph{time(7)} im Abschnitt \emph{The Epoch}
bzw.\\ \emph{http://de.wikipedia.org/wiki/Unixzeit}), wobei sie dazu pro Tag einen geeigneten
Repräsentanten finden sollten. Sinnvolle Funktionen um damit umzugehen sind \emph{time(2)} und,
falls notwendig, \emph{mktime(3)}. \\
Bitte beachten Sie auch die Allgemeinen Hinweise zu Beispielgruppe 3 und die Richtlinien
für die Erstellung von C-Programmen.\\
Vergessen Sie keinesfalls auf sprechende Fehlermeldungen.

\end{document}

