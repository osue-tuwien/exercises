\input{../../template.ltx}
\begin{document}

\osuetitle{3}

\section*{Aufgabenstellung -- ProcDB}

In dieser Aufgabe setzen Sie eine Prozessdatenbank in C um. Die Implementierung
soll aus zwei Programmen bestehen: einem Server, der die Prozessdatenbank hält
und Anfragen über deren Inhalt bearbeitet, und einem Client, mit welchem der
Benutzer Informationen aus der Prozessdatenbank vom Server abfragen kann. Die
Kommunikation zwischen den Prozessen soll mittels Shared Memory realisiert
werden und die Synchronisierung über Semaphore erfolgen.

Der Client soll dem Benutzer ein Interface bieten, mit dem der Benutzer über
Befehle Informationen vom Server abfragen kann. Diese Information soll eines
der Felder \emph{cpu}, \emph{mem}, \emph{time} oder \emph{command}, welche
jeder Eintrag in der Prozessdatenbank enthält, sein. Sie soll zu einem
bestimmten Prozess, der durch seine PID gegeben ist, oder dem Prozess, bei
welchem dieses Feld den höchsten oder den niedrigsten Wert in der Datenbank
hat, abgefragt werden können.

\subsection*{Anleitung}

Die Kommunikation zwischen den Clients und dem Server soll mittels einem
einzigen Shared Memory Object erfolgen (\textbf{nicht} einem pro Client).  Es
darf auch nicht die gesamte Prozessdatenbank im Shared Memory geladen sein,
sondern nur die Information, die der Server mit einem einzigen Client
austauscht.  Allerdings muss eine beliebige Anzahl von Clients gleichzeitig und
unabhängig voneinander mit dem Server kommunizieren können. Insbesondere darf
das Warten auf Input eines Clients nicht andere Clients blockieren.

Server und Client sollen Freigabe der Shared Memory Objects und Semaphore
koordinieren. Spätestens wenn der Server und alle Clients terminiert haben,
müssen alle Semaphore und Shared Memory Objects freigeben sein.

\subsubsection*{Server}
\begin{verbatim}
USAGE: procdb-server input-file
\end{verbatim}

Der Server legt zu Beginn die benötigten Resourcen an. Anschließend liest er
die als Argument angegebene Eingabedatei ein, parst die Information (z.B.\
mittels \texttt{fgets}, \texttt{strtok}, \texttt{sscanf}) und legt sie in einer
geeigneten Datenstruktur (z.B.\ Array, Liste, Hashtabelle) im Speicher ab.
Diese dient dem Server im Weiteren als Informationsquelle für das Bearbeiten
der Requests der Clients.

Anschließend bearbeitet der Server Anfragen von Clients (siehe "`Client"'). Der
Server sucht den entsprechenden Eintrag in der Datenbank und sendet dem Client
eine Antwort mit der gewünschten Information.

%Sollte der Server keinen Eintrag
%mit der gegebenen PID in der Datenbank finden, so schickt er dem Client eine
%Fehlernachricht.

Der Server soll durch die Signale \texttt{SIGINT} und \texttt{SIGTERM} zum
Terminieren gebracht werden. Dabei soll er "`sauber"' beenden, also alle
verwendeten Resourcen (z.B. dynamisch reservierter Speicher, Semaphore, Shared
Memory) freigeben. Das soll auch im Fehlerfall passieren.


\subsubsection*{Client}
\begin{verbatim}
USAGE: procdb-client
\end{verbatim}


Der Client verbindet sich beim Starten zuerst mit dem Server ("`verbinden"'
bedeutet hier zu überprüfen, ob die Semaphore und das Shared Memory Object,
welche zur Kommunikation notwendig sind, existieren). Anschließend nimmt er vom
Benutzer Befehle auf der Standardeingabe zum Abfragen von Informationen vom
Server entgegen. Ein Befehl hat das Format

\begin{verbatim}
  pid field
\end{verbatim}

wobei \emph{pid} entweder ein numerischer Wert zur Auswahl eines bestimmten
Prozesses anhand der PID ist, oder eine Zeichenkette aus \{\texttt{"{}min"},
\texttt{"{}max"}, \texttt{"{}sum"}, \texttt{"{}avg"}\}.
\emph{Field} ist eine Zeichenkette, die die entsprechende Information
zurückliefern soll:

\begin{itemize}
  \item \texttt{"{}cpu"{}}: die CPU-Auslastung, ausgegeben in Prozent, auf
    Zehntelprozent ("{}0.1\%"{}) genau.
  \item \texttt{"{}mem"{}}: der Speicherverbrauch, ausgegeben in Prozent, auf
    Zehntelprozent genau.
  \item \texttt{"{}time"{}}: die Zeitdauer, über welche der Prozess bereits
    läuft, ausgegeben in Stunden, Minuten und Sekunden
  \item \texttt{"{}command"{}}: das Kommando, mit welchem der Prozess
    gestartet wurde, als String.
\end{itemize}

Nur für die numerischen Felder können anstatt einer konkreten PID oben genannte
Zeichenketten angegeben werden, welche den niedrigsten Wert, den höchsten Wert,
die Summe, oder den Durchschnitt abfragen.
%
Sobald der Client die Antwort erhält, gibt er die PID des Prozesses (oder
\texttt{"{}-"{}} im Fall von Summe oder Durchschnitt) und den Wert
wie beschrieben aus.
%
Ungültige Abfragen sollen schon beim Client behandelt werden und für sie
soll kein Request erzeugt werden.
(Eine Anfrage für eine nicht existierende PID bildet keine ungültige Abfrage
- dies muss der Server prüfen und eine entsprechende Antwort generieren.)
%
Die Befehlseingabe endet mit \emph{EOF} (Ctrl-D auf dem Terminal).
Wie beim Server ist auch beim Client auf eine saubere Terminierung (auch bei
Signalen und im Fehlerfall) zu achten.



\subsection*{Datenformat}

\subsubsection*{Datenbank}

Die Datenbank-Datei ist im CSV (comma-separated values) gegeben.
Jeder Eintrag (Zeile) enthält die Information eines Prozesses, welcher
aus folgenden, durch Komma getrennte Felder enthält:

\begin{itemize}
  \item \emph{pid}: Die Prozess ID.
  \item \emph{cpu}: Die CPU-Auslastung in Promille (Zehntelprozent).
  \item \emph{mem}: Der Speicherverbrauch in Promille.
  \item \emph{time}: Die Laufzeit des Prozesses in Sekunden.
  \item \emph{command}: Das Kommando als Zeichenkette, höchstens 56
    Zeichen lang
\end{itemize}

\paragraph{Beispiel:}

\begin{verbatim}
2214,0,0,1392074,/usr/bin/ssh-agent /usr/bin/dbus-launch --exit-with-sess
2230,2,35,390837,/usr/lib/thunderbird/thunderbird
9270,0,0,1124541,vim README.txt
14141,24,66,89729,/usr/lib/firefox/firefox
\end{verbatim}



\subsubsection*{Shared Memory}

Es soll eine geeignete Struktur definiert werden, die Felder für die Anfrage,
als auch für die Antwort enthält. Die Art der Information, die abgefragt wird
(cpu, mem, time, command) soll als Aufzählungstyp definiert sein.





\osueguidelinesthree

\end{document}
