% TeX source file
% Sysprog WS 2005
% Beispiel 2: chstat
% Dietmar Schabus (e0147322@student.tuwien.ac.at)

\documentclass{article}
\usepackage[german]{babel}
\usepackage{a4wide}
\parindent0pt

\begin{document}

\begin{center}
\begin{Large}
OPERATING SYSTEMS UE BEISPIEL 3
\end{Large}
\end{center}




\section*{Aufgabenstellung}

Implementieren Sie zwei Programme {\tt readin} und {\tt chstat}, die
zu einem Eingabetext eine Buchstabenstatistik erstellen.

\begin{verbatim}
SYNOPSIS:
     readin
     chstat [-v]
\end{verbatim} 




\section*{Anleitung}

Das Programm {\tt readin} liest von {\tt stdin} und "ubergibt jede
gelesene Zeile "uber ein Shared Memory an das Programm {\tt chstat},
welches die Zeile liest und zu jedem Buchstaben speichert, wie oft er
vorgekommen ist. Gro"s-/Kleinschreibung soll dabei nicht beachtet
werden, es wird also zwischen A und a nicht unterschieden. Weiters
k"onnen Sie davon ausgehen, dass nur ASCII-Zeichen vorkommen, Umlaute
und dergleichen m"ussen Sie also nicht behandeln.

{\tt chstat} gibt dann eine Liste der Buchstaben A bis Z aus, jeweils
mit der Anzahl der Vorkommnisse sowie einer prozentuellen Angabe der
H"aufigkeit des Buchstabens (abgerundet auf ganze Prozent). Alle
anderen Zeichen werden in einer Kategorie ``andere''
zusammengefasst. Zus"atzlich wird die Gesamtanzahl der gelesenen
Zeichen ausgegeben.

Beispiel:
\begin{verbatim}
       A: 12    15%
       B: 6     7%
       C: 8     10%
       .
       .
       .
       Z: 0     0%
  andere: 24    30%
  gesamt: 80    100%
\end{verbatim}

Wird {\tt chstat} mit der Option {\tt -v} aufgerufen, gibt es nach
jeder gelesenen Zeile die aktualisierte Statistik aus, ohne {\tt -v}
wird nur die endg"ultige Statistik "uber alle Zeilen ausgegeben.

{\tt chstat} soll die n"achste Zeile erst dann lesen, wenn {\tt
readin} sie vollst"andig ins Shared Memory geschrieben hat und {\tt
readin} soll erst dann die n"achste Zeile ins Shared Memory schreiben,
wenn die vorherige schon von {\tt chstat} gelesen wurde.

Durch die Eingabe von {\tt EOF} (also durch Dr"ucken von {\tt Ctrl-D})
wird das Ende des Textes angezeigt und beide Programme terminieren.



\section*{Testen}

Die Option {\tt -v} von {\tt chstat} eignet sich gut zum interaktiven
Testen, da die Auswirkungen einer neuen Zeile auf die Statistik sofort
sichtbar sind, das Testen ohne {\tt -v} ist f"ur die Umleitung von
{\tt stdin} auf ein Testfile vielleicht sinnvoller:
\begin{verbatim}
  readin < mytestfile
\end{verbatim}

Bitte beachten Sie auch die Allgemeinen Hinweise zur Beispielgruppe 3
und die Richtlinien f"ur die Erstellung von C-Programmen.

\end{document}
