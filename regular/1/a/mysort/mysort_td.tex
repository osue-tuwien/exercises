\input{../../../template.ltx}

\begin{document}

\osuetitle{1}

\section*{Aufgabenstellung A -- mysort}
Schreiben Sie eine abgewandelte Version des \textsc{Unix}-Kommandos
\osueprog{sort} als C-Programm. Es soll nur die Option \osuearg{-r} (absteigend
sortieren) implementiert werden. Alle anderen Optionen können ignoriert werden.
Wird keine Datei angegeben, so sollen die Daten über \osueglvar{stdin}
eingelesen werden.

\begin{verbatim}
    SYNOPSIS
        mysort [-r] [file1] ...
\end{verbatim}

\subsection*{Anleitung}
Lesen Sie alle Dateien zeilenweise in einen dafür geeigneten Puffer ein.
Sortieren Sie danach die Daten mit Hilfe von \osuefunc{qsort(3)}. Im Anschluss
geben Sie die nun sortieren Daten auf \osueglvar{stdout} aus. Es kann davon
ausgegangen werden, dass keine Zeile länger als 1022 Zeichen ist (ohne Newline).

\subsection*{Testen}
Testen Sie Ihr Programm mit verschieden Eingaben. Erstellen Sie zum Beispiel
eine Testdatei \osuefilename{t1} mit folgenden Zeilen:

\begin{osuefmtcode}
  Priority9 cat
  Priority2 ls
  Priority7 cat mysort.h
\end{osuefmtcode}

Rufen Sie Ihr Programm dann wie folgt auf:

\begin{osuefmtcode}
$ \osueinput{./mysort < t1}
  Priority2 ls
  Priority7 cat mysort.h
  Priority9 cat

$ \osueinput{cat t1 \textup{|} ./mysort}
  Priority2 ls
  Priority7 cat mysort.h
  Priority9 cat

$ \osueinput{./mysort -r t1 t1}
  Priority9 cat
  Priority9 cat
  Priority7 cat mysort.h
  Priority7 cat mysort.h
  Priority2 ls
  Priority2 ls
\end{osuefmtcode}

\subsection*{Hinweis}
\osuecmd{{ ( echo a ; echo B )} | sort} sortiert \verb+B+ vor \verb+a+, weil
die Großbuchstaben im Zeichensatz vor den Kleinbuchstaben kommen.
\osuecmd{{ ( echo a ; echo B )} | sort -f} würde diese Eigenschaft ignorieren.
Die Option \osuearg{-f} muss aber \textbf{nicht} implementiert werden.

\osueguidelinesone

\end{document}
