\input{../../../template.ltx}

\begin{document}

\osuetitle{1}

\section*{Aufgabenstellung A}

Eines kalten Herbstabends finden Sie am Bildschirm eine seltsame Botschaft, die
Ihnen erklärt, dass Sie auserwählt wurden.

Beeindruckt beschließen Sie den Anweisungen darin sogleich Folge zu leisten --
bevor Sie überhaupt noch wissen, worum es geht. Der Absender behauptet wichtige
Informationen für Sie zu haben, benötigt aber schnellstens ein Tool, um Ihnen
die geheime, möglicherweise existenzverändernde Botschaft zuschicken zu können.
Sie erkennen sofort die Wichtigkeit der Situation und machen sich aufgeregt an
die Implementierung des Programms \osueprog{stegit}, um raschest die geheime
Botschaft lesen zu können.

\begin{verbatim}
    SYNOPSIS:
        stegit -f|-h [-o <filename>]

         -f                 find mode
         -h                 hide mode
        [-o <filename>]     output filename
\end{verbatim}

Dieses Programm soll in zwei Modi ausgeführt werden können: \emph{Verstecken}
und \emph{Finden}.

Im Modus \emph{Verstecken} wird solange von \osueglvar{stdin} gelesen, bis
\osueconst{EOF} auftritt oder die Enter-Taste betätigt wird. Dann wird die
eingelesene Botschaft in einem Text versteckt und auf \osueglvar{stdout}
oder in die Ausgabedatei ausgegeben. Im Modus \emph{Finden} soll wiederum von
\osueglvar{stdin} so lange gelesen werden, bis \osueconst{EOF} auftritt.
Aus dem eingelesenen Text soll die geheime Botschaft ermittelt werden und
entweder auf \osueglvar{stdout} angezeigt oder in die Ausgabedatei
geschrieben werden. Falls eine Ausgabedatei angegeben wurde, muss dafür gesorgt
werden, dass sie überschrieben wird, wenn sie schon vorhanden war; ansonsten
soll sie neu angelegt werden.

\subsection*{Anleitung}
Suchen Sie sich 28 beliebige Wörter und initialisieren Sie damit ein
Zeichenketten-Array. Dieses soll sich fix im Programm befinden und nicht
veränderbar sein. Mit Hilfe der Array-Indizes weisen Sie jedem Buchstaben im
englischen Alphabet, sowie den Sonderzeichen \emph{Leerzeichen} und
\emph{Punkt}, ein Wort zu. Damit sollte es kein Problem mehr sein, die
eingelesene Botschaft in (unter Umständen) unsinnigen Text umzuwandeln.
Sonderzeichen und nicht zugeordnete Zeichen werden ignoriert (geschluckt). Damit
der resultierende End-Text "`realistischer"' aussieht, bauen Sie zufällig (nach
je 5 bis 15 Wörtern) Punkte ein -- die Sie beim Einlesen im Find-Modus natürlich
wiederum ignorieren.  Die maximale Länge der geheimen Botschaft (Cleartext)
können Sie auf 300 Zeichen begrenzen.

Beispieltabelle:

\begin{osuefmtcode}
p -> "der"
s -> "Himmel"
c -> "ist"
h -> "heute"
t -> "klar"
. -> "la"
\end{osuefmtcode}

Die Geheimbotschaft:

\begin{osuefmtcode}
pscht...
\end{osuefmtcode}

würde nach der Hide-Operation in folgenden Text resultieren (der Punkt ist
Zufall):

\begin{osuefmtcode}
der himmel ist heute klar. la la la
\end{osuefmtcode}

\subsection*{Testen}

Starten Sie das Programm:

\begin{osuefmtcode}
./stegit -h -o secret_message_within
\end{osuefmtcode}

und geben Sie folgenden Text ein (möglichst ohne dass Sie mitlesen!):

\begin{osuefmtcode}
es muessen beide aufgabenstellungen a und b geloest werden
\end{osuefmtcode}

Drücken Sie die Enter-Taste und betrachten Sie die nun versteckte Botschaft mit:

\begin{osuefmtcode}
cat secret_message_within
\end{osuefmtcode}

Versuchen Sie aus dem entstandenen Text mit:

\begin{osuefmtcode}
./stegit -f < secret_message_within
\end{osuefmtcode}

die geheime Botschaft wiederherzustellen. Sie könnte Sie vor bösen
Überraschungen bewahren.

\osueguidelinesone

\end{document}
