\input{../../../template.ltx}

\begin{document}

\osuetitle{1}

\section*{Aufgabenstellung A}

Implementieren Sie ein Programm {\tt ispalindrom}, welches
eingegebene Strings auf Palindromeigenschaften überprüft.

\begin{verbatim}
    SYNOPSIS:
        ispalindrom [-s] [-i]
\end{verbatim}

Das Programm {\tt ispalindrom} soll zeilenweise Strings mit einer
Länge von bis zu 40 echten Zeichen von der Standardeingabe lesen;
prüfen, ob ein Palindrom vorliegt, d.h. der Text rückwärts gelesen
mit sich selbst ident ist; und den Text gefolgt von "`ist ein
Palindrom"' bzw. "`ist kein Palindrom"' auf die Standardausgabe
ausgeben. Die Option {\tt -s} soll bewirken, dass Leerzeichen
ignoriert werden; die Option {\tt -i} soll bewirken, dass nicht
zwischen Groß- und Kleinschreibung unterschieden wird.

\subsection*{Testen}

Testen Sie Ihr Programm mit verschiedenen Eingaben, wie z.B.:

\begin{verbatim}
      % ispalindrom
      Reliefpfeiler
      Reliefpfeiler ist kein Palindrom
      reliefpfeiler
      reliefpfeiler ist ein Palindrom
      % ispalindrom -i -s
      Reliefpfeiler
      Reliefpfeiler ist ein Palindrom
      O Genie der Herr ehre Dein Ego
      O Genie der Herr ehre Dein Ego ist ein Palindrom
      xxxxxxxxxxxxxxxxxxxxxxxxxxxxxxxxxxxxxxxxxxxxxxxxxxxx
      ispalindrom: Eingabe zu lang, max 40 Zeichen!
\end{verbatim}

\osueguidelinesone

\end{document}
