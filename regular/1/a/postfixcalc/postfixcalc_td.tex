\input{../../../template.ltx}

\begin{document}

\osuetitle{1}

\section*{Aufgabenstellung A -- postfixcalc}
Schreiben Sie einen Rechner, der die Postfix-Notation versteht.
Implementieren Sie die vier Grundrechenarten (\(+\), \(-\), \(*\), \(/\)) sowie
die Winkelfunktionen Sinus und Cosinus. Die Option \osuearg{-i} erzwingt eine
ganzzahlige Ausgabe. Wird \osuearg{-a} angegeben, ist nach der Berechnung der
Betrag (Absolutwert) auszugeben. Die maximale Zeilenlänge ist auf 1024 Zeichen
beschränkt.

\begin{verbatim}
    SYNOPSIS
        calc [-i] [-a] [file1 [file2 ...]]
\end{verbatim}

Postfix ist eine Notation, die ohne Klammerung auskommt und daher für Maschinen
leicht lesbar ist. Im Gegensatz zur Infix-Notation wird die Operation immer nach
den Operanden angegeben. Dazu einige Beispiele:

\begin{tabular}{ll}
\toprule
Infix           & Postfix \\
\midrule
\(1+2\)         & \(1 \: 2 \: +\) \\
\(2*(1+3)\)     & \(2 \: 1 \: 3 + *\) \\
\(\sin(1/5)\)   & \(1 \: 5 \: / \sin\) \\
\((1+2)*(3+4)\) & \(1\:2 + 3\:4 + *\) \\
\bottomrule
\end{tabular}

\subsection*{Anleitung}
Lesen Sie aus der Datei bzw.\ den Dateien zeilenweise bis zum Erreichen von
\osueglvar{EOF}. Wenn keine Datei angegeben ist, soll von \osueglvar{stdin}
gelesen werden. Da jede Zeile einer Rechnung entsprechen soll, können Sie die
Berechnung sofort ausführen.

Verwenden Sie einen Stack um die Ergebnisse zu berechnen. Extrahieren Sie
wiederholt mit \osuefunc{strtod(3)} eine Zahl aus der Zeile und legen Sie
diese oben auf den Stack. Konnte keine Zahl konvertiert werden, überprüfen Sie,
ob ein korrekter Operator angegeben wurde. In diesen Fall nehmen Sie zwei Zahlen
(für die binären Operatoren \osueinput{+}, \osueinput{-}, \osueinput{/} und
\osueinput{*}) bzw.\ eine Zahl (für die unären Operatoren \osueinput{s} und
\osueinput{c}) vom Stack und führen die entsprechende Berechnung durch. Das
Ergebnis soll dann wiederum auf den Stack gelegt werden.
Wenn das Zeilenende erreicht wurde, darf sich nur noch ein Element -- das
Ergebnis -- am Stack befinden.

\subsection*{Testen}
Testen Sie Ihr Programm mit verschieden Eingaben. Erstellen sie zum Beispiel
eine Testdatei \osuefilename{t1} mit folgenden Zeilen:

\begin{osuefmtcode}
  2 3 +
  2 3 * 4 -
  3 2 s* -4.5 +
  1.3E-2 100 *
\end{osuefmtcode}

Rufen Sie Ihr Programm dann wie folgt auf:

\begin{osuefmtcode}
$ \osueinput{./calc t1}
  5.000000
  2.000000
 -1.772108
  1.300000

$ \osueinput{./calc < t1}
  5.000000
  2.000000
 -1.772108
  1.300000

$ \osueinput{./calc -i -a t1}
  5
  2
  1
  1
\end{osuefmtcode}

\subsection*{Hinweis}
Leerzeichen zwischen Operatoren und Zahlen sind erlaubt, aber nicht notwendig. \verb|1 2 -3 + | ist nicht gültig, da das Minus zur \verb|3| gehört und von \osuefunc{strtod(3)} konsumiert wird. \verb|1 2 *3 + | oder \verb|1 2 - 3 + | hingegen ist in Ordnung.
Für die Winkelfunktionen müssen Sie \osuefilename{math.h} einbinden, sowie beim Linken den Parameter \osuearg{-lm} verwenden.

\osueguidelinesone

\end{document}
