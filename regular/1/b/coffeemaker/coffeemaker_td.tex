\input{../../../template.ltx}

\begin{document}
\osuetitle{1}

\section*{Aufgabenstellung B -- Coffee Maker}\label{sec:aufgabenstellung}

Am Institut für Technische Informatik wurde eine Kaffeemaschine mit einem
Raspberry Pi mit Netzwerkanbindung und automatischer Kaffeekapsel-Versorgung
erweitert. Nun soll ein Service ``Kaffeeproduktion'' entwickelt werden.

\subsection*{Anleitung}
Implementieren Sie einen Client und einen Server, die mittels TCP/IP
miteinander kommunizieren. Angestellte des Instituts starten den Client um eine
Anfrage an die Kaffeemaschine zu senden. Der Server soll auf der Kaffeemaschine
Anfragen bearbeiten und die Kaffeeproduktion einleiten.

Nachdem eine Verbindung zwischen dem Client und dem Server hergestellt wurde,
wird die Anfrage übermittelt. Angestellte können Tassengröße, Stärke und
Geschmacksrichtung des Kaffees wählen. Die Kaffeemaschine soll mit
Produktionsdauer oder Status antworten.

Verwenden Sie geeignete Ausgaben (auf \osueglvar{stdout}) am Server und Client
um Anfragen und Kaffeeproduktion folgen zu können. Alle Fehlermeldungen müssen
auf \osueglvar{stderr} ausgegeben.

\subsubsection*{Server}
Der Server soll auf eingehende Verbindungen warten. Sobald eine Verbindung
akzeptiert wurde, wird die Anfrage bearbeitet, d.h. Kaffee produziert. Die
Kaffeeproduktion soll mit einem Verzögerung (delay) simuliert werden. Verwenden
Sie \emph{Tassengröße in ml} $\cdot \, 10 \, ms$ (zB: bei einer Tassengröße von
100ml soll 1s lang gewartet werden). Neue Verbindungen werden erst zugelassen,
wenn die Kaffeeproduktion der letzten Anfrage beendet wurde.

Der Server verwaltet den aktuellen Status der Kaffeemaschine, welcher nach
jeder Kaffeeproduktion aktualisiert werden muss. Initial soll der Wassertank
voll und die Kapselkammer, die gebrauchte Kapseln sammelt, leer sein. Pro
Anfrage: soll je nach Tassengröße der Inhalt des Wassertanks verringert werden
und die Anzahl der Kapseln in der Kapselkammer inkrementiert werden. Ist das
Wasser für die aktuelle Anfrage nicht mehr ausreichend oder die Kapselkammer
voll soll ein Fehlercode an den Client zurück gesendet werden. Kann der Kaffee
produziert werden, wird die Dauer der Produktion in Millisekunden
zurückgeschickt.

Sobald der Server eines der Signale \osueconst{SIGINT} oder
\osueconst{SIGTERM} empfängt, soll der Socket geschlossen und das
Programm mit Rückgabewert 0 beendet werden.

\begin{verbatim}
SYNOPSIS
    server [-p PORT] [-l WATER] [-c CUPS]
\end{verbatim}

Die Größe des Wassertanks (default: 2l) und der Kapselkammer (default: 10
Kapseln) kann beim Starten des Servers ausgewählt werden. Die Portnummer ist
per default 1234.

\subsubsection*{Client}
Legen Sie zuerst einen TCP/IP-Socket an. Stellen Sie dann die zum Hostnamen des
Servers zugehörige IP-Adresse fest und verbinden Sie sich mit dem
Server. Unmittelbar nach Verbindungsaufbau wird die Anfrage zur
Kaffeeproduktion übermittelt und die Antwort des Servers ausgewertet und
angezeigt. Danach soll der Socket geschlossen und das Programm beendet werden.

Da mehrere Angestellte gleichzeitig auf den Service zugreifen können und die
Kaffeeproduktion ihre Zeit braucht, kann es sein, dass der Client etwas warten
muss bis dieser an der Reihe ist. Sollte der Angestellte zu ungeduldig sein,
kann der Client mit den Signalen \osueconst{SIGINT} oder \osueconst{SIGTERM}
beendet werden. Der Socket soll daraufhin geschlossen und das Programm mit
Rückgabewert 0 beendet werden.

\begin{verbatim}
SYNOPSIS
    client [-h HOSTNAME] [-p PORT] SIZE STRENGTH FLAVOUR
\end{verbatim}

Dem Client können Hostname (default: \emph{localhost}) und Portnummer (default:
\emph{1234}) übergeben werden. Die Tassengröße (\osueconst{SIZE}), Stärke
(\osueconst{STRENGTH}) und Geschmacksrichtung (\osueconst{FLAVOUR}) des Kaffees
sollen über die Argumente gesetzt werden.

\subsection*{Protokoll}
\label{sec:prot}
Der Client übermittelt Tassengröße, Stärke und Geschmacksrichtung. Folgende
Werte sollen unterstützt werden:
%
\begin{description}
\item[Tassengröße:] 25ml, 100ml, 250ml
\item[Stärke:] leicht, mittel, stark, extra-stark
\item[Geschmacksrichtung:] Decaffeinato, Kazaar, Volluto, Ciocattino, Vanilio
\end{description}

Der Server empfängt die Nachricht des Clients. Die Antwort des Servers ist vom
Status der Kaffeemaschine (Wassertank, Kapselkammer) abhängig.
%
\begin{description}
\item[Status OK] (Kaffee kann produziert werden) $\rightarrow$ Antwort: Dauer
  der Kaffeeproduktion. Die Dauer wird durch die Tassengröße (Wassermenge)
  bestimmt (siehe Beschreibung des Servers).
\item[Status NOK] (Kaffeemaschine muss gewartet werden) $\rightarrow$ Antwort:
  Fehlercode. Der Client soll zwischen Wassertank leer und Kapselkammer voll
  unterscheiden können.
\end{description}

Definieren Sie ein geeignetes Format der Nachrichten. Stellen Sie sicher, dass
die Übertragung unabhängig von der Architektur funktioniert.


\osueguidelinesone

\osueadvertise{Wollen Sie den Client erweitern und auf das Robot Operating
  System\footnote{\url{http://www.ros.org/}} portieren?}

\end{document}
