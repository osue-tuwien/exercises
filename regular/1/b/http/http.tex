\input{../../../template.ltx}

\begin{document}
\osuetitle{1}

\section*{Assignment B -- HTTP}
\label{sec:intro}

Implement a client and a server for the Hypertext Transfer Protocol (HTTP).

HTTP is text-based request-response protocol.

\subsection*{Instructions}

Write a client program and a server program
which partially implement version 1.1 of the HTTP.

\subsubsection*{Client}

The client takes a URL as input,
connects to the server specified in that URL
and requests the file specified in the URL.

\begin{verbatim}
SYNOPSIS
    client [-p PORT] [ -o FILE | -d DIR ] URL
EXAMPLE
    client en.wikipedia.org/wiki/Hypertext_Transfer_Protocol
\end{verbatim}

The client may be called with a number of options:
\begin{itemize}
\item Option \texttt{-p} can be used to specify the port
on which the client shall attempt to connect to the server.
If this option is not used the port defaults to 80.
\item
\end{itemize}

The client creates a TCP/IP socket
and connects to the hostname specified in the URL.
For simplicity you may assume that
the hostname is the part before any of the following characters:

\texttt{;} \texttt{/} \texttt{?} \texttt{:} \texttt{@} \texttt{=} \texttt{\&}

Once the connection is established,
the client sends a request for the file specified in the URL
(the filename is the part starting at the first \texttt{/} )
using the HTTP method \texttt{GET}.
The request header must start with the request line
and as a minimum must contain the \texttt{Host} field.
For instance, the minimal request header for the URL
\texttt{en.wikipedia.org/index.html} is:
\begin{verbatim}
GET /index.html HTTP/1.1
Host: en.wikipedia.org
\end{verbatim}

After transmitting his request, the client waits for a response from the server.
Your client must correctly parse the response header.

If the response header is invald,
the client prints the message \texttt{"Protocol error!"}
and terminates with exit code 2.

If the response status in the first line of the response header is not 200,
the client prints a message with the response status
and terminates with exit code 3.

\subsubsection*{Server}

The server waits for connections from clients
and transmits the requested files.

\begin{verbatim}
SYNOPSIS
    server [-p PORT] [-i INDEX] DOC_ROOT
EXAMPLE
    server -p 1280 -i index.html ~/Documents/my_website/
\end{verbatim}


The server must support the HTTP method \texttt{GET}.
Support for any other method is optional.
Your server must be able to correctly parse any fields in the request header.
Fields which are unknown to the server are silently ignored.


If your server receives an invalid request

If the server cannot locate and open the file identified by the URL,
then sends a response header with the status code 404
and closes the connection.

Otherwise, the server sends a response header with status code 200
and at minimum following fields (with the values changed accordingly):
\begin{verbatim}
HTTP/1.1 200 OK
Date: Mon, 23 May 2005 22:38:34 GMT
Content-Type: text/html; charset=UTF-8
Content-Encoding: UTF-8
Content-Length: 138
Last-Modified: Wed, 08 Jan 2003 23:11:55 GMT
Connection: close
\end{verbatim}
Once the header is transmitted,
the server proceeds with the transmission of the requested file.

\subsection*{Protocol}
\label{sec:prot}

\subsection*{Bonus Points}

\textbf{Bonus points are only awarded
if both your client and your server program are FULLY FUNCTIONAL
and COMPLY WITH ALL INSTRUCTIONS!
Do not try to implement any of the bonus tasks
while your programs do not function properly!
When adding features for bonus tasks,
check that the basic requirements are still fulfilled,
otherwise you risk loosing more points than you might gain!}

\subsubsection*{Server}

\begin{itemize}

\item \textbf{Compression (2):}
Implement compression of the transferred content in the gzip format.
Make use of the zlib compression library (see sect.~``\nameref{sect:zlib}'')
for compressing the data.

2 bonus points are awarded if your server recognizes clients
which accept data encoded in gzip format,
responds to requests from these clients with the line
\begin{verbatim}
Content-Encoding: gzip
\end{verbatim}
in the responde header
and correctly compresses the content in the body into that format.

\item \textbf{Reuse connection (2):}
Allow a client to make multiple requests using the same connection.
If a client indicates that he wishes to reuse the connection with the line
\begin{verbatim}
Connection: keep-alive
\end{verbatim}
in the request header,
the server sends a response with the same line in the response header
and once all content has been transfered,
the server waits for another request from that client.
A \textbf{timeout must be implemented},
which closes the connection
if the client does not send another request within 15 seconds.

\item \textbf{Simultaneous connections (2):}
Use a file descriptor set (see the man page \texttt{fd\_set(3)})
to manage connections from multiple clients in parallel.

2 bonus points are awarded if your server uses \texttt{select(3p)} or \texttt{pselect(3p)}
to monitor all open connections simultaneously,
accepts new connections instantaneously while other connections are kept open
and responds to a request on any connection instantaneously.

Do not forget to set a large enough \texttt{backlog} when calling \texttt{listen(2)}.

Your server must never repeatedly poll connections for new data
or use any kind of busy waiting!

\end{itemize}

\subsubsection*{Client}

\begin{itemize}

\item \textbf{Decompression (2):}
Implement decompression of content in the gzip format.
Make use of the zlib compression library (see sect.~``\nameref{sect:zlib}'')
for decompressing the data.

2 bonus points are awarded if your client advertises this capability with the line
\begin{verbatim}
Accept-Encoding: gzip
\end{verbatim}
in the request header and is able to correctly decompress content sent with that encoding.

\item \textbf{Load external content (2):}
Load external content from \texttt{<link>}, \texttt{<script>} and \texttt{<img>} tags.

If the transferred content is of content type \texttt{text/html},
then search for \texttt{<link>}, \texttt{<script>} and \texttt{<img>} tags
(this can be implemented with a simple search
for the strings \verb|"<link"|, \verb|"<script"| and \verb|"<img"| in the content).
If any of these tags contains a \texttt{rel} or \texttt{src} attribute
(which must come before the \texttt{>} at the end of the tag),
then the file specified by this attribute is also requested.

\end{itemize}

\subsubsection*{Using zlib}
\label{sect:zlib}

zlib\footnote{\url{http://zlib.net/}} is installed in the TILab.
You can use it by including the \texttt{zlib.h} header:
\begin{verbatim}
#include <zlib.h>
\end{verbatim}
Also, you will have to link the zlib library by adding \texttt{-lz} to your linker options.

\textbf{It would be a shame if you loose points because using zlib produces
compiler or linker warnings or errors.
Make sure to TEST YOUR IMPLEMENTATION IN THE TILAB!}

\osueguidelinesone

\end{document}
